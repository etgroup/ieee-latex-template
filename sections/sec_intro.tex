
%!TEX root = ../article.tex

% Introduction
\section{Introduction}
\label{sec:introduction}

This is a well organized \LaTeX  template for the IEEE papers that can be used by  the graduate students to prepare the IEEE conference and journal papers.



\subsection{Titles}

One can modify the author list, institution, address, correspondence email and title of paper in the file \verb+variables.tex+ and \verb+article.tex+.


\subsection{Keywords and abstract}
The keywords and abstract can be modified in the files \verb+keywords.tex+ and \verb+abstract.tex+, respectively.


\subsection{Acronym}

Prior to use the acronyms, one needs to add the acronym into the file \verb+/acronym/acronym.tex+.
To refer an acronym, use the \LaTeX  command \verb+\gls{}+.

This is an example in how to referring an acronym \gls{IEEE}.

\gls{ECT}, \gls{ERT} and \gls{EMT} are three different electrical tomography modalities \ldots


\subsection{Math formula}

Some frequently used formulas can be found in the file \verb+math.tex+.
To refer a formula, use the corresponding command.
For example, we can write the Maxwell equations in

\begin{equation}\label{eq:maxwell}
  \maxwell
\end{equation}

And, the signal-to-noise ratio can be expressed as:

\begin{equation}\label{eq:snr}
  \snr
\end{equation}

Also, one can use the image correlation coefficient to evaluate the image quality, \emph{i.e.},


\begin{equation}\label{eq:corr}
  \corrcoe
\end{equation}

\subsection{Bibliography}

We can use the Jabref software to manage the bibliography.
The file \verb+bib.bib+ can be used as a common library to add references to the manuscript.

\begin{itemize}
  \item Add BibTeX Entry: On the website (https://ieeexplore.ieee.org), click the button ``Download citations'' and choose output format ``BibTeX''. Copy the text and paste it into the file \verb+bib.bib+ by using Jabref (New BibTeX entry, BibTeX source).    
  \item Merge Multiple Bibliography files: In Jabref, Import into current library, Deselect all duplicates, Save. 
  \item Insert Citations: This is an example in how to cite a bibliography entry \cite{cao2018, cui2019}.
\end{itemize}


